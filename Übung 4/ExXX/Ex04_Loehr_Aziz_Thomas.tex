% LaTeX Template für Abgaben an der Universität Stuttgart
% Autor: Sandro Speth
% Bei Fragen: Sandro.Speth@iste.uni-stuttgart.de
%-----------------------------------------------------------
% Hauptmodul des Templates: Hier können andere Dateien eingebunden werden
% oder Inhalte direkt rein geschrieben werden.
% Kompiliere dieses Modul um eine PDF zu erzeugen.

% Dokumentenart. Ersetze 12pt, falls die Schriftgröße anzupassen ist.
\documentclass[12pt]{scrartcl}
% Einbinden der Pakete, des Headers und der Formatierung.
% Mit den \include und \input Befehlen können Dateien eingebunden werden:
% \include: Fügt einen Seitenumbruch nach dem Text ein
% \input: Fügt KEINEN Seitenumbruch nach dem Text ein
\input{../styles/Packages.tex}
\input{../styles/FormatAndHeader.tex}

% Counter für das Blatt und die Aufgabennummer.
% Ersetze die Nummer des Übungsblattes und die Nummer der Aufgabe
% den Anforderungen entsprechend.
% Definiert werden die Counter in FormatAndHeader.tex
% Beachte:
% \setcounter{countername}{number}: Legt den Wert des Counters fest
% \stepcounter{countername}: Erhöht den Wert des Counters um 1.
\setcounter{sheetnr}{1} % Nummer des Übungsblattes
\setcounter{exnum}{1} % Nummer der Aufgabe

% Beginn des eigentlichen Dokuments
\begin{document}
% Nutze den \exercise{Aufgabenname} Befehl, um eine neue Aufgabe zu beginnen.
% Möchtest du eine Aufgabe in der Nummerierung überspringen, schreibe vor der Aufgabe: \stepcounter{exnum}
% Möchtest du die Nummer einer Aufgabe auf eine beliebige Zahl x setzen, schreibe vor der Aufgabe: \setcounter{exnum}{x}

\section*{1 Predicate Logic}
\subsection*{1.1 First Order Logic}
\subsubsection*{Definitions}
\begin{itemize}
    \item $P(x)$: $x$ is a person
    \item $F(x)$: $x$ is an ice cream flavor
    \item $L(x, y)$: $x$ loves $y$
    \item $S(x, y)$: $x$ shaves $y$
    \item $R(x)$: $x$ is a reindeer
    \item $N(x)$: $x$ has a red nose
    \item $W(x)$: $x$ is weird
    \item $C(x)$: $x$ is a clown
    \item $b$: The Barber (constant)
    \item $r$: Rudolph (constant)
\end{itemize}

\subsubsection*{Solutions}

\begin{enumerate}
    \renewcommand{\labelenumi}{\textbf{\alph.}} 
    \item \textbf{a.    There is an ice cream flavor loved by everyone.}
    \[\exists x (F(x) \land \forall y (P(y) \rightarrow L(y, x)))\]

    \item \textbf{b.    It is not true that everyone loves some ice cream flavor.}
    \[
    \neg \forall x (P(x) \rightarrow \exists y (F(y) \land L(x, y)))
    \]

    \item \textbf{c.    Anyone who does not shave himself must be shaved by the barber.}
    \[
    \forall x (P(x) \land \neg S(x, x) \rightarrow S(b, x))
    \]

    \item \textbf{d.    Whomever the barber shaves, must not shave himself.}
    \[
    \forall x (P(x) \land S(b, x) \rightarrow \neg S(x, x))
    \]

    \item \textbf{e.    Rudolph is a reindeer and Rudolph has a red nose.}
    \[
    R(r) \land N(r)
    \]

    \item \textbf{f.    Anyone with a red nose is weird or is a clown.}
    \[
    \forall x (N(x) \rightarrow (W(x) \lor C(x)))
    \]

    \item \textbf{g.    No reindeer is a clown.}
    \[
    \forall x (R(x) \rightarrow \neg C(x))
    \]
    
\end{enumerate}
\subsection*{1.2 Clausal Forms }
\begin{enumerate}
    \renewcommand{\labelenumi}{\textbf{\alph.}}


    \item \textbf{a.    \forall x \forall y (R(x, y) \rightarrow (R(x, y) \land Q(y)))}
    
    Step 1: Eliminate implication ($A \rightarrow B \equiv \neg A \lor B$)
    \[ \forall x \forall y (\neg R(x, y) \lor (R(x, y) \land Q(y))) \]
    
    Step 2: Distribute $\lor$ over $\land$
    \[ \forall x \forall y ((\neg R(x, y) \lor R(x, y)) \land (\neg R(x, y) \lor Q(y))) \]
    
    Step 3: Simplify ($\neg A \lor A \equiv \text{True}$)
    \[ \forall x \forall y (\text{True} \land (\neg R(x, y) \lor Q(y))) \]
    \[ \forall x \forall y (\neg R(x, y) \lor Q(y)) \]
    
    Step 4: Drop universal quantifiers
    \[ \neg R(x, y) \lor Q(y) \]

    \textbf{Resulting Clause:}
    \[ \{ \neg R(x, y), Q(y) \} \]

    % --- Problem B ---
    \item \textbf{b.   \forall x \exists y \forall z (P(x, y, z) \rightarrow \exists u R(x, u, z))}

    Step 1: Eliminate implication
    \[ \forall x \exists y \forall z (\neg P(x, y, z) \lor \exists u R(x, u, z)) \]

    Step 2: Prenex Normal Form (move quantifiers left)
    \[ \forall x \exists y \forall z \exists u (\neg P(x, y, z) \lor R(x, u, z)) \]

    Step 3: Skolemization
    \begin{itemize}
        \item Replace $y$ with $f(x)$ (depends on $\forall x$)
        \item Replace $u$ with $g(x, z)$ (depends on $\forall x$ and $\forall z$)
    \end{itemize}
    \[ \forall x \forall z (\neg P(x, f(x), z) \lor R(x, g(x, z), z)) \]

    Step 4: Drop universal quantifiers
    \[ \neg P(x, f(x), z) \lor R(x, g(x, z), z) \]

    \textbf{Resulting Clause:}
    \[ \{ \neg P(x, f(x), z), R(x, g(x, z), z) \} \]


    \item \textbf{c.    \forall x (\neg \exists y P(x, y) \land \neg (Q(x) \land \neg R(x)))}

    Step 1: Move negations inward
    \[ \forall x (\forall y \neg P(x, y) \land (\neg Q(x) \lor \neg \neg R(x))) \]
    \[ \forall x (\forall y \neg P(x, y) \land (\neg Q(x) \lor R(x))) \]

    Step 2: Standardize variables and move quantifiers
    \[ \forall x \forall y (\neg P(x, y) \land (\neg Q(x) \lor R(x))) \]

    Step 3: Drop universal quantifiers and split conjunction
    \[ \neg P(x, y) \land (\neg Q(x) \lor R(x)) \]

    \textbf{Resulting Clauses:}
    \begin{align*}
        1.\ & \{ \neg P(x, y) \} \\
        2.\ & \{ \neg Q(x), R(x) \}
    \end{align*}

\end{enumerate}

\subsection*{1.3 Knowledge Extraction via Resolution}
\subsubsection*{Conversion to Clause Form (CNF)}

Convert the Knowledge Base sentences $S1-S4$ and the negated goal into a set of clauses.

\begin{itemize}
    \item Sentence S1: $\forall t \forall x (Observe(t, x) \land Danger(x) \rightarrow Suggestion(t, \text{flee}))$
    \\ Elim. Implication: $\neg Observe(t, x) \lor \neg Danger(x) \lor Suggestion(t, \text{flee})$
    \\ Clause $C_1$: $\{ \neg Observe(t, x), \neg Danger(x), Suggestion(t, \text{flee}) \}$

    \item Sentence S2: $\forall t (\neg \exists x (Observe(t, x) \land Danger(x)) \rightarrow Suggestion(t, \text{stay}))$
    \\ Elim. Implication: $\exists x (Observe(t, x) \land Danger(x)) \lor Suggestion(t, \text{stay})$
    \\ Skolemization ($x \to h(t)$): $(Observe(t, h(t)) \land Danger(h(t))) \lor Suggestion(t, \text{stay})$
    \\ Clause $C_{2a}$: $\{ Observe(t, h(t)), Suggestion(t, \text{stay}) \}$
    \\ Clause $C_{2b}$: $\{ Danger(h(t)), Suggestion(t, \text{stay}) \}$

    \item Sentence S3: $Danger(\text{lion})$
    \\ Clause $C_3$: $\{ Danger(\text{lion}) \}$

    \item Sentence S4: $Observe(\text{now}, \text{lion})$
    \\ Clause $C_4$: $\{ Observe(\text{now}, \text{lion}) \}$

    \item Goal: Prove $\exists x Suggestion(\text{now}, x)$.
    \\ Negated Goal with Answer Literal: $\neg Suggestion(\text{now}, z) \lor Ans(z)$
    \\ Clause $C_G$: $\{ \neg Suggestion(\text{now}, z), Ans(z) \}$
\end{itemize}

\subsubsection*{Resolution Trace}

\begin{itemize}
    \item Resolve $C_1$ and $C_4$:
    \\ $C_1: \{ \neg Observe(t, x), \neg Danger(x), Suggestion(t, \text{flee}) \}$
    \\ $C_4: \{ Observe(\text{now}, \text{lion}) \}$
    \\ Substitution: $\theta = \{ t/\text{now}, x/\text{lion} \}$
    \\ Resolvent $R_1$: $\{ \neg Danger(\text{lion}), Suggestion(\text{now}, \text{flee}) \}$

    \item Resolve $R_1$ and $C_3$:
    \\ $R_1: \{ \neg Danger(\text{lion}), Suggestion(\text{now}, \text{flee}) \}$
    \\ $C_3: \{ Danger(\text{lion}) \}$
    \\ Substitution: $\theta = \emptyset$
    \\ Resolvent $R_2$: $\{ Suggestion(\text{now}, \text{flee}) \}$

    \item Resolve $R_2$ and $C_G$:
    \\ $R_2: \{ Suggestion(\text{now}, \text{flee}) \}$
    \\ $C_G: \{ \neg Suggestion(\text{now}, z), Ans(z) \}$
    \\ Substitution: $\theta = \{ z/\text{flee} \}$
    \\ Resolvent $R_3$: $\{ Ans(\text{flee}) \}$
\end{itemize}

\subsubsection*{Conclusion}
The clause contains only the answer literal. Thus, the Knowledge Base entails the query with the answer:
\[ \mathbf{x = \text{flee}} \]

\section*{2 Uncertainty}
\subsection*{2.1 Probability}

\subsection*{2.2 Naive Bayes’ Classifier}

\end{document}
