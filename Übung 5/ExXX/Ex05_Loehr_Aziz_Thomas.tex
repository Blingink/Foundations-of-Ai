% LaTeX Template für Abgaben an der Universität Stuttgart
% Autor: Sandro Speth
% Bei Fragen: Sandro.Speth@iste.uni-stuttgart.de
%-----------------------------------------------------------
% Hauptmodul des Templates: Hier können andere Dateien eingebunden werden
% oder Inhalte direkt rein geschrieben werden.
% Kompiliere dieses Modul um eine PDF zu erzeugen.

% Dokumentenart. Ersetze 12pt, falls die Schriftgröße anzupassen ist.
\documentclass[12pt]{scrartcl}
% Einbinden der Pakete, des Headers und der Formatierung.
% Mit den \include und \input Befehlen können Dateien eingebunden werden:
% \include: Fügt einen Seitenumbruch nach dem Text ein
% \input: Fügt KEINEN Seitenumbruch nach dem Text ein
\input{../styles/Packages.tex}
\input{../styles/FormatAndHeader.tex}

% Counter für das Blatt und die Aufgabennummer.
% Ersetze die Nummer des Übungsblattes und die Nummer der Aufgabe
% den Anforderungen entsprechend.
% Definiert werden die Counter in FormatAndHeader.tex
% Beachte:
% \setcounter{countername}{number}: Legt den Wert des Counters fest
% \stepcounter{countername}: Erhöht den Wert des Counters um 1.
\setcounter{sheetnr}{1} % Nummer des Übungsblattes
\setcounter{exnum}{1} % Nummer der Aufgabe

% Beginn des eigentlichen Dokuments
\begin{document}
% Nutze den \exercise{Aufgabenname} Befehl, um eine neue Aufgabe zu beginnen.
% Möchtest du eine Aufgabe in der Nummerierung überspringen, schreibe vor der Aufgabe: \stepcounter{exnum}
% Möchtest du die Nummer einer Aufgabe auf eine beliebige Zahl x setzen, schreibe vor der Aufgabe: \setcounter{exnum}{x}

\section*{1 Probabilistic Reasoning}

\section*{2 Making Decisions}
\subsection*{2.1 Utility Theory (1)}
The lottery over A and B that is indifferent to C must be of the form [p, A; 1-p, B]. It must also satisfy this equation to be indifferent to C:
$p*U(A) + (1-p)*U(B) = U(C) $\\
That can be resolved as $p*455+(1-p)*(-150) = 50$ and finally $p=\frac{200}{605}= 0,331$. 
The lottery over A and B is therefore [0,331, A; 0,669, B].

\subsection*{2.2 Utility Theory (2)}
The given lotteries are [0,5, B; 0,5, C] and [1, A] = A. The expected Utilities of these are: $U([0,5, B; 0,5, C]) = 0,5*U(B) + 0,5*U(C) = 0,5*20 + 0,5*0 = 10$ and $U(A) = 5$. 
Therefore the lottery [0,5, B; 0,5, C] is preferred to the lottery [1, A].

\subsection*{2.3 Markov Decision Processes (MDPs)}
\subsubsection*{2.3.1 Computing discount factor $\gamma$}

\end{document}
