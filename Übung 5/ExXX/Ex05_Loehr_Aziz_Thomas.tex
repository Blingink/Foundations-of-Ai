% LaTeX Template für Abgaben an der Universität Stuttgart
% Autor: Sandro Speth
% Bei Fragen: Sandro.Speth@iste.uni-stuttgart.de
%-----------------------------------------------------------
% Hauptmodul des Templates: Hier können andere Dateien eingebunden werden
% oder Inhalte direkt rein geschrieben werden.
% Kompiliere dieses Modul um eine PDF zu erzeugen.

% Dokumentenart. Ersetze 12pt, falls die Schriftgröße anzupassen ist.
\documentclass[12pt]{scrartcl}
% Einbinden der Pakete, des Headers und der Formatierung.
% Mit den \include und \input Befehlen können Dateien eingebunden werden:
% \include: Fügt einen Seitenumbruch nach dem Text ein
% \input: Fügt KEINEN Seitenumbruch nach dem Text ein
\input{../styles/Packages.tex}
\input{../styles/FormatAndHeader.tex}

% Counter für das Blatt und die Aufgabennummer.
% Ersetze die Nummer des Übungsblattes und die Nummer der Aufgabe
% den Anforderungen entsprechend.
% Definiert werden die Counter in FormatAndHeader.tex
% Beachte:
% \setcounter{countername}{number}: Legt den Wert des Counters fest
% \stepcounter{countername}: Erhöht den Wert des Counters um 1.
\setcounter{sheetnr}{1} % Nummer des Übungsblattes
\setcounter{exnum}{1} % Nummer der Aufgabe

% Beginn des eigentlichen Dokuments
\begin{document}
% Nutze den \exercise{Aufgabenname} Befehl, um eine neue Aufgabe zu beginnen.
% Möchtest du eine Aufgabe in der Nummerierung überspringen, schreibe vor der Aufgabe: \stepcounter{exnum}
% Möchtest du die Nummer einer Aufgabe auf eine beliebige Zahl x setzen, schreibe vor der Aufgabe: \setcounter{exnum}{x}

\section*{1 Probabilistic Reasoning}

\section*{2 Making Decisions}
\subsection*{2.1 Utility Theory (1)}
The lottery over A and B that is indifferent to C must be of the form [p, A; 1-p, B]. It must also satisfy this equation to be indifferent to C:
$p*U(A) + (1-p)*U(B) = U(C) $\\
That can be resolved as $p*455+(1-p)*(-150) = 50$ and finally $p=\frac{200}{605}= 0,331$. 
The lottery over A and B is therefore [0,331, A; 0,669, B].

\subsection*{2.2 Utility Theory (2)}
The given lotteries are [0,5, B; 0,5, C] and [1, A] = A. The expected Utilities of these are: $U([0,5, B; 0,5, C]) = 0,5*U(B) + 0,5*U(C) = 0,5*20 + 0,5*0 = 10$ and $U(A) = 5$. 
Therefore the lottery [0,5, B; 0,5, C] is preferred to the lottery [1, A].

\subsection*{2.3 Markov Decision Processes (MDPs)}
\subsubsection*{2.3.1 Computing discount factor $\gamma$}
Since only continuing in state 4 yields a reward, the optimal policy is to continue in every state until state 5 is reached.
The  optimal value of state 1 is therefore: $U^*(s_1) = R(s_1, a_C) + \gamma R(s_2, a_C) + \gamma^2 R(s_3, a_C) + \gamma^3 R(s_4, a_C) + \gamma^4 R(s_5, a_C) = 1$ \\
Adding the rewards results in: $U^*(s_1) = 0 + 0 + 0 + \gamma^3 * 10 + 0 = 10* \gamma^3$ = 1. \\
$\gamma$ is therefore $\sqrt[3]{\frac{1}{10}} \approx 0,464$.

\subsubsection*{2.3.2 Markov Decision Process - Value Iteration}
The utility values of the grid world are initialized as 0 for every state, except for the terminal states (3,4) and
(2,4) which have fixed utilities of +1 and -1 respectively. This is shown in the figure below on the left side. 
After the first iteration these values change as shown in the figure below on the right side. They are calculated as follows: \\
$U_1((3,3)) = \max_{a \in A} \sum_{s'} P(s'|(3,3),a) * [R((3,3),a,s') + \gamma * U_0(s')]  = 0,8*(-0,2+0,9*1)+0,1*(-0,2+0,9*0)+0,1*(-0,2+0,9*0) = 0,52$ \\
$U_1((2,3)) = \max_{a \in A} \sum_{s'} P(s'|(2,3),a) * [R((2,3),a,s') + \gamma * U_0(s')]  = 0,8*(-0,2+0,9*0)+0,1*(-0,2+0,9*-1)+0,1*(-0,2+0,9*0) = -0,29$ \\
$U_1((1,4))$ is calculated analogously to $U_1((2,3))$ and results in -0,29. \\
$U_1((1,1)) = \max_{a \in A} \sum_{s'} P(s'|(1,1),a) * [R((1,1),a,s') + \gamma * U_0(s')]  = 0,8*(-0,2+0,9*0)+0,1*(-0,2+0,9*0)+0,1*(-0,2+0,9*0) = -0,2$ \\
All other states are calculated analogously to $U_1((1,1))$ and result to -0,2.\\
\begin{center}
    \includegraphics[width=0.9\textwidth]{fai ex05 2.3.2.jpeg}
\end{center}
The utility value of state (3,3) for the second iteration can now be calculated as follows: \\
$U_2((3,3)) = \max_{a \in A} \sum_{s'} P(s'|(3,3),a) * [R((3,3),a,s') + \gamma * U_1(s')]  = 0,8*(-0,2+0,9*1)+0,1*(-0,2+0,9*0,52)+0,1*(-0,2+0,9*-0,29) = 0,56 + 0,0268 - 0,0461 = 0,5407$ \\
\end{document}
