% LaTeX Template für Abgaben an der Universität Stuttgart
% Autor: Sandro Speth
% Bei Fragen: Sandro.Speth@iste.uni-stuttgart.de
%-----------------------------------------------------------
% Hauptmodul des Templates: Hier können andere Dateien eingebunden werden
% oder Inhalte direkt rein geschrieben werden.
% Kompiliere dieses Modul um eine PDF zu erzeugen.

% Dokumentenart. Ersetze 12pt, falls die Schriftgröße anzupassen ist.
\documentclass[12pt]{scrartcl}
% Einbinden der Pakete, des Headers und der Formatierung.
% Mit den \include und \input Befehlen können Dateien eingebunden werden:
% \include: Fügt einen Seitenumbruch nach dem Text ein
% \input: Fügt KEINEN Seitenumbruch nach dem Text ein
\input{../styles/Packages.tex}
\input{../styles/FormatAndHeader.tex}

% Counter für das Blatt und die Aufgabennummer.
% Ersetze die Nummer des Übungsblattes und die Nummer der Aufgabe
% den Anforderungen entsprechend.
% Definiert werden die Counter in FormatAndHeader.tex
% Beachte:
% \setcounter{countername}{number}: Legt den Wert des Counters fest
% \stepcounter{countername}: Erhöht den Wert des Counters um 1.
\setcounter{sheetnr}{1} % Nummer des Übungsblattes
\setcounter{exnum}{1} % Nummer der Aufgabe

% Beginn des eigentlichen Dokuments
\begin{document}
% Nutze den \exercise{Aufgabenname} Befehl, um eine neue Aufgabe zu beginnen.
% Möchtest du eine Aufgabe in der Nummerierung überspringen, schreibe vor der Aufgabe: \stepcounter{exnum}
% Möchtest du die Nummer einer Aufgabe auf eine beliebige Zahl x setzen, schreibe vor der Aufgabe: \setcounter{exnum}{x}

\section*{1 Adversarial Search and Games}
\subsection*{1.1 Minimax Search}
\subsubsection*{a) Compute Minimax-Value}
\begin{center}
    \includegraphics[width=\textwidth]{FAI 3.1.1.jpeg}
\end{center}

\subsection*{b) Independency of value from x and y}
The leafs of x and y don't have any impact on the root node's value, because of the structure of the tree.
To change the value of the tree node, a child of the root must have a higher value than 9, otherwise thats the chosen max value.
But the left child of the root will always evaluate to 4, if the parent of x and y doesn't have a lower value than that.
To have an impact on the root node, x or y would have to be a value which is greater than 9 und lower than 4, which is impossible.

\subsection*{c) Pruning with optimal move order}
\begin{center}
    \includegraphics[width=0.9\textwidth]{FAI 3.1.3.jpeg} 
\end{center}
Using this order will result in 3 nodes being pruned, with two of them being leaf nodes.

\subsection*{d) Pruning with worst move order}
\begin{center}
    \includegraphics[width=0.9\textwidth]{FAI 3.1.4.jpeg} 
\end{center}
Using this order, none of the nodes can be pruned.

\subsection*{1.2 Games of Chance }
\subsubsection*{a) Resulting Search Tree}
\begin{center}
    \includegraphics[width=0.9\textwidth]{Exxx/FAI 3.2.png} 
\end{center}

\subsubsection*{b) Expected Minimax Value}
y1s1 -> Y = -800    S = -666.67
\\y1s2 ->             S = 0
\\Therefore, if Y choses y1, then S will choose s2.
\\y2s1 -> Y = 0       S = -25
\\y2s2 ->             S = 0
\\If, Y choses y2, then S will likely choose s1, as there is a possibility that Y will land in Park Lane in the next round, hence gaining much more.
\\Then, Y = -333.33
\\For Y, y1 gives -800, while y2 gives -333.33. So y2 is selected.
\\Therefore, y2s1
\\Expected minimax value = -333.33 - (-25) = -308.33

\subsubsection*{c) Number of Visits and Upper Confidence Bound}
To find: the number of visits($N$) and the UCB1 value for each node.
\\The UCB1 formula for each node $i$ is:
\[UCB1_i = V_i + C\sqrt{lnN_p/N_i}\]
where,
\\$V_i$: average observed outcome
\\$n_p$: total visit count of parent node
\\$n_i$: total visit count of node $i$
\\Path \quad Outcomes \qquad \qquad $N_{path}$ \quad $V_{path}$(Average)
\\$y_1s_1$ \quad -1250, -1300, -1175 \quad  3  \qquad     -1241.67
\\$y_1s_2$ \quad -800,-800, -800  \qquad \quad 3 \qquad      -800
\\$y_2s_1$ \quad -400, +50    \qquad \qquad \quad      2   \qquad    -175
\\$y_2s_2$ \quad -50, -75     \qquad \qquad  \qquad     2   \qquad    -62.5
\\$N_y$ = 10
\\Intermediate node values and visits
\\Node $S_1$:   $V_{S_1} = -800$    $N_{S_1} = 6$
\\Node $S_2$:   $V_{S_2} = -62.5$   $N_{S_2} = 4$
\\Node $Y$:   $V_{Y} = -582.5$      $N_{Y} = 10$
\\$UCB_{y_1} = -799.38$
\\$UCB_{y_2} = -61.74$
\begin{center}
    \includegraphics[width=0.7\textwidth]{Exxx/FAI 3.jpg} 
\end{center}

\subsection*{1.3 Correctness of $\alpha-\beta$-pruning}
\subsubsection*{a)}
For a max-node the true minimax value is the maximum over all children. If the best value seen so far is $\gamma$, then the final minimax value 
val($n_0$) satisfies,
\[ val(n_0) \ge \gamma \]

\subsubsection*{b)}
For each node $n_i$ its siblings are denoted \{$n_{i1}$,.....,$n_{ib_i}$\} and the parent relation is $n_i$ \in children($n_{i-1}$). The type (min/max) alternates along the path.
\\If $n_i$ is a min-node:
\[ n_i = min(n_{i+1}, n_{i1}, n_{i2}, ....., n_{ib_i})\]
If $n_i$ is a max-node:
\[ n_i = max(n_{i+1}, n_{i1}, n_{i2}, ....., n_{ib_i})\]

\subsubsection*{c)}
If $n_i$ is a min-node:
\[ n_i = min(l_i, n_{i+1}, r_i)\]
If $n_i$ is a max-node:
\[ n_i = max(l_i, n_{i+1}, r_i)\]

\subsubsection*{d)}
Because $n_1$ is a min-node, one immediate upper bound on the final $n_1$ comes from its already-seen left-children, $l_1$.
\[ n_i = min(l_i, (other children))\]
So if the value contributed by the path is $\le$ $l_1$, then the minimum will remain $l_1$ and the path cannot lower $n_1$.
\\Because the path alternates min/max and $n_j$ is a max-nod, the min-ancestors along the path are exactly 1,3,5,...,j-1.
\\Therefore if $n_j$ is to affect $n_1$, then:
\[ n_j < min(l_1, l_3, l_5, ...., l_{j-1})\]

\section*{2 Propositional Logic}
\subsection*{2.1 Who is lying?}

\textbf{We define the variables:}
\begin{itemize}
    \item $J$: John tells the truth.
    \item $P$: Peter tells the truth.
    \item $E$: Emma tells the truth.
\end{itemize}
\subsubsection*{(a)}
\textbf{The statements:}  \\
\textbf{John says: 'Peter always lies.'} \\
This means $J$ is true if and only if $P$ is false.
Formula: ($J) \leftrightarrow \neg (P$) \\
\textbf{Peter says: 'Either John is a liar or Emma is a liar, but not both.'} \\
There are two possibilities for this statement to be true:
    \begin{itemize}
        \item Possibility 1: John lies ($\neg J$) AND Emma tells the truth ($E$).
        \item Possibility 2: John tells the truth ($J$) AND Emma lies ($\neg E$).
        \item We connect these two possibilities with an OR.
    \end{itemize}
Formula: $P \leftrightarrow ((\neg J \land E) \lor (J \land \neg E))$ \\
    \textbf{Emma says: 'If John is a liar, then Peter is also a liar.'} \\
    This is a classical implication.\\
Formula: $E \leftrightarrow (\neg J \rightarrow \neg P)$



\subsubsection*{(b)}
The set of formulae is:
\begin{align}
    1.\quad & J \leftrightarrow \neg P \\
    2.\quad & P \leftrightarrow \left( (\neg J \lor \neg E) \land \neg(\neg J \land \neg E) \right) \\
    3.\quad & E \leftrightarrow (\neg J \rightarrow \neg P)
\end{align}
\textbf{Goal:} Find a truth assignment that satisfies all three.

\subsubsection*{Step 1: Analysis of John’s formula}
Formula: $J \leftrightarrow \neg P$

Using the definition $A \leftrightarrow B \equiv (A \rightarrow B) \land (B \rightarrow A)$, we have:
\[
(J \leftrightarrow \neg P) \equiv (J \rightarrow \neg P) \land (\neg P \rightarrow J)
\]
Now rewrite the implications:
\begin{align*}
    J \rightarrow \neg P & \equiv \neg J \lor \neg P \\
    \neg P \rightarrow J & \equiv P \lor J
\end{align*}
So John’s statement becomes:
\[
(\neg J \lor \neg P) \land (P \lor J)
\]
From this, we see the required relation:
\[
\boxed{J = \neg P}
\]

\subsubsection*{Step 2: Analysis of Emma’s formula}
Formula: $E \leftrightarrow (\neg J \rightarrow \neg P)$

Rewrite the implication using $A \rightarrow B \equiv \neg A \lor B$:
\[
\neg J \rightarrow \neg P \equiv J \lor \neg P
\]
Now substitute John’s relation $J = \neg P$:
\[
J \lor \neg P \equiv (\neg P) \lor \neg P \equiv \neg P
\]
\[
E \leftrightarrow \neg P
\]
We now know:
\[
J = \neg P \quad \text{and} \quad E = \neg P
\]
Therefore:
\[
\boxed{J = E}
\]
\textbf{Interim result:} John and Emma must have the same truth value.

\subsubsection*{Step 3: Analysis of Peter’s formula}
Peter states his 'one lies but not both' condition:
\[
P \leftrightarrow ((\neg J \lor \neg E) \land \neg(\neg J \land \neg E))
\]

\textbf{3.1 Substitute $J = E$} \\
Since we know $J = E$, then $\neg J = \neg E$. Thus the formula becomes:
\[
P \leftrightarrow ((\neg J \lor \neg J) \land \neg(\neg J \land \neg J))
\]
Simplify step by step:
\begin{enumerate}
    \item $\neg J \lor \neg J = \neg J$
    \item $\neg J \land \neg J = \neg J$
    \item $\neg(\neg J) = J$
\end{enumerate}
Therefore Peter’s statement reduces to:
\[
P \leftrightarrow (\neg J \land J)
\]

\textbf{3.2 Simplify the contradiction} \\
The term $(\neg J \land J)$ is a contradiction, meaning it is always false:
\[
\neg J \land J \equiv \text{false}
\]
So:
\[
P \leftrightarrow \text{false}
\]
Thus:
\[
\boxed{P = \text{false}}
\]

\subsubsection*{Step 4: Determine J and E}
We previously established:
\[
J = \neg P \quad \text{and} \quad E = \neg P
\]
Since $P = \text{false}$, we conclude:
\begin{align*}
    J &= \neg \text{false} = \text{true} \\
    E &= \neg \text{false} = \text{true}
\end{align*}

\subsection*{Final answer}
\begin{itemize}
    \item \textbf{John tells the truth}
    \item \textbf{Peter lies}
    \item \textbf{Emma tells the truth}
\end{itemize}

\subsection*{2.2 Knowledge Bases}
We are given the knowledge base:
\[
    K=\{\,A\vee(B\vee\neg C),\; A\Leftrightarrow B,\; (C\land A)\Rightarrow D\,\}.
\]
\subsubsection*{(a)}
1. The formula \(A\vee(B\vee\neg C)\) simplifies to:
\[
    A \vee B \vee \neg C.
\]
2. The equivalence \(A\Leftrightarrow B\) is rewritten as:
\[
    (A\rightarrow B)\land(B\rightarrow A),
\]
which becomes:
\[
    (\neg A \vee B)\ \land\ (\neg B \vee A).
\]

3. The implication \((C\land A)\Rightarrow D\) becomes:
\[
    \neg(C\land A)\vee D = \neg C \vee \neg A \vee D.
\]

Thus, the CNF of the entire knowledge base is:
\[
    (A\vee B\vee\neg C)\ \land\ (\neg A\vee B)\ \land\ (\neg B\vee A)\ \land\ (\neg C\vee\neg A\vee D).
\]

\subsubsection*{(b)}
The clauses in the CNF are:
\[
    \{A\vee B\vee \neg C,\ \neg A\vee B,\ \neg B\vee A,\ \neg C\vee \neg A\vee D\}.
\]

As sets of literals:
\[
    \big\{\{A,B,\neg C\},\ \{\neg A,B\},\ \{\neg B,A\},\ \{\neg C,\neg A,D\}\big\}.
\]

\subsubsection*{(c)}

A definite clause contains exactly one positive literal.

Checking each clause:
\begin{itemize}
    \item \(A\vee B\vee \neg C\): two positive literals → not definite.
    \item \(\neg A\vee B\): one positive literal \(B\) → definite.
    \item \(\neg B\vee A\): one positive literal \(A\) → definite.
    \item \(\neg C\vee \neg A\vee D\): one positive literal \(D\) → definite.
\end{itemize}
Thus, the definite clauses are:
\[
    \neg A\vee B,\qquad \neg B\vee A,\qquad \neg C\vee \neg A\vee D.
\]

\subsection*{2.3 Models}
Consider the four propositions $A, B, C,$ and $D$.
The total number of possible truth assignments (interpretations) for 4 binary variables is $2^4 = 16$.
We need to find how many of these 16 assignments make the following formulae true.

\subsubsection*{(a) $B \lor \neg C$}
This formula depends only on $B$ and $C$. The variables $A$ and $D$ are 'don't cares' (irrelevant to the truth value, but must be counted).
\begin{itemize}
    \item \textbf{Analyze $B \lor \neg C$:}
    A disjunction is FALSE only if both parts are false.
    \begin{itemize}
        \item $B$ is False ($0$)
        \item $\neg C$ is False $\Rightarrow C$ is True ($1$)
    \end{itemize}
    Combinations for $(B, C)$:
    \begin{itemize}
        \item $(0, 0) \rightarrow 0 \lor 1 = 1$ (True)
        \item $(0, 1) \rightarrow 0 \lor 0 = 0$ (False)
        \item $(1, 0) \rightarrow 1 \lor 1 = 1$ (True)
        \item $(1, 1) \rightarrow 1 \lor 0 = 1$ (True)
    \end{itemize}
    So, there are \textbf{3} valid assignments for the pair $(B, C)$.

    \item \textbf{Account for $A$ and $D$:}
    For each valid combination of $B$ and $C$, the variables $A$ and $D$ can be either True or False ($2 \times 2 = 4$ variations).

    \item \textbf{Calculation:}
    \[ \text{Models} = 3 \times 2^2 = 3 \times 4 = 12 \]
\end{itemize}

\textbf{Answer: 12 models.}

\subsubsection*{(b) $A \land \neg(\neg B \lor C) \land D$}

This is a conjunction. For the formula to be True, \textbf{every} part must be True.

\begin{itemize}
    \item \textbf{Analyze the components:}
    \begin{itemize}
        \item Part 1: $A$ must be \textbf{True} ($1$).
        \item Part 2: $D$ must be \textbf{True} ($1$).
        \item Part 3: $\neg(\neg B \lor C)$ must be \textbf{True}.
    \end{itemize}

    \item \textbf{Simplify Part 3 (De Morgan's Law):}
    \[ \neg(\neg B \lor C) \equiv \neg(\neg B) \land \neg C \equiv B \land \neg C \]
    For this to be True:
    \begin{itemize}
        \item $B$ must be \textbf{True} ($1$).
        \item $C$ must be \textbf{False} ($0$).
    \end{itemize}

    \item \textbf{Calculation:}
    We have fixed values for all four variables:
    \[ A=1, \quad B=1, \quad C=0, \quad D=1 \]
    There is only 1 specific assignment that satisfies this.
\end{itemize}

\textbf{Answer: 1 model.}

\subsubsection*{(c) $((B \rightarrow D) \lor (A \rightarrow D)) \land C$}
This is a conjunction of a complex term and $C$.
\begin{itemize}
    \item \textbf{Constraint on $C$:}
    Since it is a conjunction ($\dots \land C$), $C$ must be \textbf{True} ($1$).
    This reduces our search space. We effectively assume $C=1$ and look for satisfying assignments of $A, B, D$ ($2^3 = 8$ possibilities).

    \item \textbf{Analyze the complex term:}
    Formula: $(B \rightarrow D) \lor (A \rightarrow D)$

    Rewrite implication ($X \rightarrow Y \equiv \neg X \lor Y$):
    \[ (\neg B \lor D) \lor (\neg A \lor D) \]

    Simplify (Associativity and Idempotence of $\lor$):
    \[ \neg A \lor \neg B \lor D \lor D \equiv \neg A \lor \neg B \lor D \]

    This formula is a disjunction. It is \textbf{True} in all cases \textit{except} when all literals are False.

    \item \textbf{Find the failing case:}
    The expression $\neg A \lor \neg B \lor D$ is False if:
    \begin{itemize}
        \item $\neg A$ is False $\Rightarrow A = 1$
        \item $\neg B$ is False $\Rightarrow B = 1$
        \item $D$ is False $\Rightarrow D = 0$
    \end{itemize}
    There is exactly \textbf{1} failing combination for $(A, B, D)$ out of 8 possibilities.

    \item \textbf{Calculation:}
    \begin{itemize}
        \item Total combinations for $(A, B, D)$: $8$
        \item Failing combinations: $1$
        \item Valid combinations: $8 - 1 = 7$
    \end{itemize}
    Since $C$ is fixed to True, we do not multiply further.
\end{itemize}

\textbf{Answer: 7 models.}

\end{document}
