% LaTeX Template für Abgaben an der Universität Stuttgart
% Autor: Sandro Speth
% Bei Fragen: Sandro.Speth@iste.uni-stuttgart.de
%-----------------------------------------------------------
% Hauptmodul des Templates: Hier können andere Dateien eingebunden werden
% oder Inhalte direkt rein geschrieben werden.
% Kompiliere dieses Modul um eine PDF zu erzeugen.

% Dokumentenart. Ersetze 12pt, falls die Schriftgröße anzupassen ist.
\documentclass[12pt]{scrartcl}
% Einbinden der Pakete, des Headers und der Formatierung.
% Mit den \include und \input Befehlen können Dateien eingebunden werden:
% \include: Fügt einen Seitenumbruch nach dem Text ein
% \input: Fügt KEINEN Seitenumbruch nach dem Text ein
\input{../styles/Packages.tex}
\input{../styles/FormatAndHeader.tex}

% Counter für das Blatt und die Aufgabennummer.
% Ersetze die Nummer des Übungsblattes und die Nummer der Aufgabe
% den Anforderungen entsprechend.
% Definiert werden die Counter in FormatAndHeader.tex
% Beachte:
% \setcounter{countername}{number}: Legt den Wert des Counters fest
% \stepcounter{countername}: Erhöht den Wert des Counters um 1.
\setcounter{sheetnr}{1} % Nummer des Übungsblattes
\setcounter{exnum}{1} % Nummer der Aufgabe

% Beginn des eigentlichen Dokuments
\begin{document}
% Nutze den \exercise{Aufgabenname} Befehl, um eine neue Aufgabe zu beginnen.
% Möchtest du eine Aufgabe in der Nummerierung überspringen, schreibe vor der Aufgabe: \stepcounter{exnum}
% Möchtest du die Nummer einer Aufgabe auf eine beliebige Zahl x setzen, schreibe vor der Aufgabe: \setcounter{exnum}{x}

\section*{1) True or False?}

\section*{4 Crossword Puzzles}
\subsection*{a)}
The problem of fitting words into an empty crossword puzzle could be defined as a classical search problem like this:
\begin{enumerate}
    \item state space: a crossword grid with any number of word gaps filled
    \item initial state: completely empty crossword grid
    \item actions: choosing an item from a list of words with the right length and possibly characters on specific positions if required
    \item transition model: fill the corresponding gap with the chosen word
    \item goal state: completely filled crossword grid
\end{enumerate}
A fitting search algorithm could be Stochastic Beam Search, because you may need to try to fill the grid from a previous state if no words for a given gap can be found.
A heuristic function for this problem could be the number of filled in gaps of the grid.
\subsection*{b)}
As a constraint satisfaction problem, it could look like this:
\begin{enumerate}
    \item variables: the chosen word for a specific gap
    \item domains: the given list of words which can be chosen. Can be restricted to the possible options of the corresponding gap
    \item constraints: each word has to be the corresponding length and for every overlap of two words, they must have the same character on the position of the overlap
    \item objective: number of satisfied constraints
\end{enumerate}
\end{document}
